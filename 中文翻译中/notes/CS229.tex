\documentclass[14pt,a4paper]{article}
\usepackage[utf8]{inputenc}
\usepackage{xeCJK}
\setCJKmainfont[BoldFont=SimHei]{SimSun}
\setCJKmonofont{SimSun}% 设置缺省中文字体
\usepackage{amsmath}
\usepackage{amsfonts}
\usepackage{amssymb}
\usepackage{makeidx}
\usepackage{graphicx}
\usepackage{setspace}
\usepackage[left=2cm,right=2cm,top=2cm,bottom=2cm]{geometry}
\title{CS229 Lecture notes}
\author{原作者:Andrew Ng (吴恩达),翻译:CylcesUser} 
\date{}
\begin{document}
\maketitle

\section{感知器(perception)和大型边界分类器(large margin classifiers)}  

\begin{spacing}{1.25}

\quad \quad 本章是讲义关于学习理论的最后一部分,我们介绍另外机器学习的模式。在之前的内容中,我们考虑的都是批量学习的情况,即给了我们训练样本集合用于学习,然后学习得到的假设$h$来评估和判别测试数据。在本章,我们要讲义中新的机器学习模式:在线学习,这种情况下,我们的学习算法要在进行学习的同时给出预测。学习算法获得一个样本序列,其中内容为有次序的学习样本,$(x^1,y^1),(x^2,y^2),\cdots,(x^m,y^m)$.最开始获得的就是$x^1$,然后需要预测$y^1$.在完成预测之后,在把$y^{1}$的真实值告诉算法(然后利用这个信息来进行某种学习)。接下来给算法提供$x^2$,在让算法对$y^2$进行预测,然后再把$y^2$的真实值告诉算法,这样的算法就又能学到一些信息了。这样的过程一直持续到最末尾的样本。$(x^m,y^m)$.在这种在线学习的背景下,我们关心的是算法在此过程中出错的总次数。因此,这适合需要一边学习一边给出预测的应用情形。

接下来,我们将对感知器学习算法(perceptron algorithm)的在线学习误差给出一个约束。为了让后续的推导(subsequent derivations)更容易,我们就用正负号来表征分类的标签,即假设$y\in\{-1,1\}$。回忆一下感知器算法(在第二章有讲到),其中参数$\theta\in R^{n+1}$,该算法根据下面的方程来给出预测:
$$
h(\theta)=g(\theta^{T}x) \quad(1)
$$
其中
$$
g(z)=\begin{cases}1 & if\quad z\leq 0 \\-1 & if\quad z<0\end{cases}
$$
然后,给定一个训练样本$(x,y)$,感知器学习规则(perception learning rule)就按照如下所示进行更新。如果$h_{\theta}=y$,那么不改变参数。若二者相等关系不成立,则进行更\footnote{这和之前我们看到的跟新规则的写法稍微有一点点不一样,因为这里我们把分类标签(labels)改成了$y\in\{-1,1\}$,另外学习速率参数(learning rate parameter)$\alpha$也被省去了。这个速来参数的效果只是使用某些固定的常数来对参数$\theta$进行缩放,并不会影响生成器的行为效果。}。
$$\theta=\theta+yx$$

当感知器算法作为在线学习算法运算的时候,每次对样本给出错误判断的时候,则更新参数,在下面的定理给出了这中情况下的在线学习误差的边界约束。要注意,下面的错误次数的约束边界与整个序列样本的个数$m$不具有特定的依赖关系(explicit dependence),和输入特征的维度$n$也无关。

定理(Block 1962 and Novikoff 1962)。设有一个样本序列:$(x^1,y^1),(x^2,y^2),(x^3,y^3),\cdots,(x^m,y^m)$。假设对于所有的$i$,都有$\parallel x^{(i)}\leq\parallel$,更近一步存在一个单位长度的向量$u(\parallel u\parallel_2=1)$对序列中的所有样本都满足$y^{(i)}\cdot(u^{T}x^{i})\ge \gamma$(例如,$u^{T}x^{i}\ge \gamma \quad if\quad y^{(i)}=1$,而$u^{T}x^{(i)}\leq \gamma$,若$y^{(i)}=-1$,则$u$就以一个宽度至少为$\gamma$的分界分开了样本数据。)而此感知器算法针对这个序列给出错误预测的综述的上限为$(D/\gamma)^2$

证明:感知器算法每次只针对出错的样本进行权重更新。设$\theta^{(k)}$为犯了第$k$个错误($k_th mistake$)的时候的权重。则$ \theta^{(1)}=0 $(因为初始权重为零),若第$k$个样本发生了错误在样本$(x^{(i)},y^{(i)})$ 则 $ g((x^{(i)})^{T}\theta^{(k)})\neq y^{i}$,也就意味着:
$$(x^{(i)})^{T}\theta^{(k)}y^{(i)}\leq 0\quad(2)$$
另根据感知器算法的定义,我们知道$\theta^{(k+1)}=\theta^{(k)}+y^{(i)}x^{(i)}$然后就得到:
$$
\begin{aligned}
(\theta^{(k+1)})^{T}u&=(\theta^{(k)})^{T}u+y^{(i)}(x^{(i)})^{T}u \\ 
&\ge (\theta^{(k)})^{T}u+\gamma
\end{aligned}
$$
利用一个简单的归纳法(straightforward inductive argument)得到:
$$(\theta^{(k+1)})^{T}u\ge k \gamma \quad(3)$$


还是根据感知器算法的定义可以得到:
$$
\begin{aligned}
\parallel\theta^{(k+1)}\parallel^2&=\parallel\theta^{(k)}+y^{(i)}x^{(i)}\parallel^2\\
&=\parallel\theta^{(k)}\parallel^2+\parallel x^{(i)}\parallel^2+2y^{(i)}(x^{(i)})^{T}\theta^{i}\\
&\leq \parallel\theta^{(k)}\parallel^2+\parallel x^{(i)}\parallel^2\\
&=\parallel \theta^{(k)}\parallel^2+D^2
\end{aligned}\quad(4)
$$
上面这个推导过程,第三步用到了等式(2)。另外这里还要使用一次简单的归纳法,上面的不等式(4)表明:
$$
\parallel\theta^{(k+1)}\parallel^2\leq kD^2 \quad(5)
$$
把上面的不等式$(3)$和不等式$(4)$结合起来:
$$
\begin{aligned}
\sqrt{k}D&\ge \parallel\theta^{(k+1)}\parallel\\
         &\ge (\theta^{(k+1)})^{T}u\\
         &\ge k\gamma
\end{aligned}$$
上面的第二个不等式是基于$u$,是一个单位长度的向量($z^{T}u=\parallel z\parallel \cdot \parallel u\parallel cos\phi\leq\parallel z\parallel \cdot \parallel u\parallel $)其中的$\phi$和向量$z$和向量$u$的夹角)。结果表明$k\leq(D/\gamma)^2$。因此,如果感知器犯了第$k$个错误,则$k\leq(D/\gamma)^2$。
\end{spacing}
\end{document}